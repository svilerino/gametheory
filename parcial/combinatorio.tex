\subsection{Estrategias ganadoras del jugador II}
Para los incisos a) y b) de este ejercicio, definiremos ciertas cosas a continuacion.

\textbf{Nota}: $\bigoplus$ es la suma nim.\\

\begin{theorem}[Teorema de Bouton]
\label{bouton}
	Una posicion $(X_1, \dots, X_n)$ en Nim, es \texttt{P-Position} $\leftrightarrow$ $X_1 \bigoplus ... \bigoplus X_n = 0$ 
\end{theorem}

\begin{theorem}
\label{particion-movidas}
	Las posiciones de un juego se particionan en \texttt{N-Position} y \texttt{P-Position}.
\end{theorem}

\begin{theorem}[Propiedad caracteristica de las posiciones]
	\label{position-characterization}
	Las posiciones de tipo \texttt{N-Position} y \texttt{P-Position} se definen recursivamente como:
	\begin{enumerate}
		\item Todas las posiciones terminales son \texttt{P-Position}.
		\item Desde toda \texttt{N-Position} \textbf{existe al menos una} jugada a una \texttt{P-Position}
		\item Desde toda \texttt{P-Position}, \textbf{todas} las jugadas son hacia \texttt{N-Position}.
	\end{enumerate}
\end{theorem}

\textbf{Idea:} Comenzando en una \texttt{N-Position}, en cada turno del jugador II, este siempre debe moverse a una posicion en la cual el jugador I vuelva a mover a una \texttt{N-Position}.\\

\begin{theorem}[Estrategia ganadora de II]
	\label{estrategia-ganadora}
	Si el jugador II comienza en una \texttt{N-Position}, tiene una estrategia ganadora.	
\end{theorem}
\begin{proof}
	Si el jugador II comienza en una \texttt{N-Position}. Por definicion(\ref{position-characterization}), existe una jugada que me lleva a una \texttt{P-Position}.
	Por definicion de \texttt{P-Position}, el jugador I va a tener que perder el juego porque dicha posicion es terminal o moverse a una \texttt{N-Position}, y nuevamente, el jugador II puede moverse a alguna \texttt{P-Position}. Este proceso se repite y es la estrategia ganadora para el jugador II.\\	
\end{proof}

\begin{corollary}
	\label{estrategia-comienza-I}
	Por definicion de \texttt{P-Position}(\ref{position-characterization}), todas las movidas posibles llevan a una \texttt{N-Position}. Si el jugador I comienza en una \texttt{P-Position}, el jugador II tambien tiene estrategia ganadora desde el momento que I le cede la \texttt{N-Position} al mover o pierde porque es una posicion terminal.
\end{corollary}

\begin{corollary}
	\label{sumanim-posiciones}
	Usando los teoremas \ref{bouton} y \ref{particion-movidas} tenemos que si $(X_1, \dots, X_k)$ es el nim inicial, entonces vale:
	\begin{itemize}
		\item La posicion inicial es una \texttt{P-Position} si $(X_1 \bigoplus \dots \bigoplus X_k) == 0$ 
		\item La posicion inicial es una \texttt{N-Position} si $(X_1 \bigoplus \dots \bigoplus X_k) \neq 0$ 
	\end{itemize}
\end{corollary}

\begin{theorem}[Equivalencia de suma nim a un bitwise exclusive or]
\label{equiv-nim-xor}
	La suma nim es equivalente a aplicar un \textbf{bitwise XOR} a los numeros a y b en base 2. Es decir, podemos definir la suma nim como $a \bigoplus b$ $ = $ $( (a)_2$ $\hat{}$ $(b)_2 )_{10}$.
\end{theorem}
\begin{proof}
	 Se desprende de considerar las tablas de definicion de la suma modulo 2 y la tabla de verdad del xor tomando 0 como falso y 1 como verdadero.
\end{proof}

\subsubsection{Inciso a}
Sabemos que:
\begin{itemize}
	\item El numero n, de fichas en la bolsa, es par
	\item El numero k, de pilas, es par 
\end{itemize}

Dado que n es par, luego de la etapa de colocacion de fichas, comienza jugando el nim el jugador I.\\
Como establecimos en \ref{estrategia-comienza-I} y \ref{sumanim-posiciones}. Si el nim inicial $(X_1, \dots, X_k)$ es tal que $(X_1 \bigoplus \dots \bigoplus X_k) == 0$, el jugador I comenzara en una \texttt{P-Position} y el jugador II tendra la estrategia ganadora mencionada en \ref{estrategia-ganadora}.\\

Solo resta encontrar una manera para que el nim inicial sume cero.\\

\textbf{Idea:} Le balanceo las pilas nim al jugador I de a pares. Luego, en el nim inicial $(X_1, \dots, X_k)$, tengo una cantidad par k de torres tales que:\\
\begin{enumerate}
	\item \textbf{Para cada pila, existe otra con la misma cantidad de elementos:} $\forall $ $ X_j $ $ \in $ $(X_1, \dots, X_k)$ $\exists$ $ X_i $ tal que $ (i \neq j) \land (X_i = X_j) $  
	\item \textbf{Ley de cancelacion de suma nim:} $ (X_i \bigoplus X_j) = 0$  
\end{enumerate}

Por lo tanto, el nim inicial que comienza jugando I es una \texttt{P-Position}. Como queriamos.

\begin{theorem}[Se puede armar un nim inicial que suma cero]
	Bajo estas hipotesis, siempre se puede llegar a un nim balanceado de a pares.
\end{theorem}

\begin{proof}
Por Induccion en cantidad de fichas:\\

Sean:
\begin{itemize}
	\item El numero n, de fichas en la bolsa, natural par no nulo.
	\item El numero k, de pilas, natural par no nulo.
	\item P(n): {Las pilas estan balanceadas(misma cantidad de elementos) de a pares}.
\end{itemize}

Dado que la cantidad de pilas es par, podemos establecer una relacion de apareo entre pares de pilas. Diremos que 2 pilas estan apareadas cuando tienen la misma cantidad de elementos. Concretamente, al final de cada par de turnos tendremos una cantidad par de pares de pilas apareadas.

\begin{itemize}
	\item \textbf{Caso base, n = 2}\\
	Como k es par(y no nulo), entonces $k \geq 2 $. El jugador I coloca una ficha en alguna de las pilas. Luego, existe alguna pila vacia. El jugador II pone una ficha en la pila vacia. Finalmente para toda pila en el nim, o bien esta vacia, o bien es alguna de las 2 pilas con 1 elemento.

	\item \textbf{Caso inductivo, n $\rightarrow$ n+1}
	Vale la hipotesis inductiva. El jugador I coloca una ficha en alguna pila $X_j$. Como las pilas estan balanceadas, existe una pila $X_i$, \texttt{apareada} a $X_j$ tal que si el jugador II coloca una ficha en ella quedan $X_j$ y $X_i$ con la misma cantidad de fichas, balanceadas nuevamente. Como no tocamos ninguna otra pila. Las pilas vuelven a quedar balanceadas.

\end{itemize}
\end{proof}

\subsubsection{Inciso b}
Sabemos que:
\begin{itemize}
	\item El numero n, de fichas en la bolsa, es impar
\end{itemize}

Dado que n es impar, juega primero el jugador II en el nim, queremos que II comience en una \textbf{N-Position}.
Equivalentemente por \ref{sumanim-posiciones}, queremos que la posicion inicial $(X_1, \dots, X_k)$ sume nim \textbf{distinto de cero}, $(X_1 \bigoplus \dots \bigoplus X_k) \neq 0$.

\begin{theorem}
Sea $(X_1, \dots, X_k)$ una posicion en el nim, esta posicion es una \texttt{P-Position} $\leftrightarrow$ $(X_1 \bigoplus \dots \bigoplus X_k) = 0$, usando \ref{equiv-nim-xor} tenemos que:\\

\vspace{0.3cm}

\begin{tabular}{|c|c c|}
			\hline
			& $X_1 	=$ & $	(X_{1,1}, \dots, X_{1,t})_2$ \\\hline
			& $X_2 	=$ & $	(X_{2,1}, \dots, X_{2,t})_2$ \\\hline
$\bigoplus$  & & $ \dots$\\\hline
			& $X_k 	=$ & $	(X_{k,1}, \dots, X_{k,t})_2$ \\\hline 
			& $0$	 & $	(0,   ..., 0)_2$ \\\hline
\end{tabular}
\vspace{0.3cm}

\textbf{Nota:} $t \in \mathbb{N}$ es la cantidad de bits requerida para representar el maximo numero del conjunto ${ X_1, \dots, X_k}$.\\

Esto ocurre unicamente cuando para todo $j$ entre $1$ y $t$ vale que $X_{1,j} + X_{2,j} + \dots + X_{k,j} = 0 $ $ (mod$ $2)$. Es decir, cuando la cantidad de unos en cada una de las columnas es par.\\
\end{theorem}

\begin{proof}
	Es facil ver esto, considerando la asociatividad del XOR en la suma por columna mencionada anteriormente, aplicando la operacion xor de a 2 elementos por columna hasta que finalmente da 0 cuando la cantidad de unos es par, 1 caso contrario.
\end{proof}

\begin{corollary}
	Como nosotros queremos comenzar a jugar el nim en una \textbf{N-Position}, debe existir \textbf{ALGUNA} columna tal que la operacion exclusive or en dicha columna de distinto de cero, es decir, que dicha columna tenga una cantidad \textbf{IMPAR} de unos.
\end{corollary}

\begin{lemma}
	El bit mas a la derecha en la suma nim especificada como operacion exclusive or, indica el bit de paridad del numero $X_j$, es decir. Si dicho bit es 0, el numero $X_j$ es par, caso contrario, el numero $X_j$ es impar.
\end{lemma}

\textbf{Idea:} Conseguir una cantidad impar de torres con cantidad impar de fichas en el armado de las torres.\\

\begin{theorem}[]
	Al ser $n$ impar, el numero de columnas $X_j$ en el nim inicial $(X_1, \dots, X_k)$ tales que $X_j$ es impar,tambien lo sera.
\end{theorem}

\begin{proof}

Sabemos que $n = X_1 + X_2 + \dots + X_n $ y n es impar.

\begin{itemize}
	\item[] Tomando congruencia modulo 2.
	\item[] $n \equiv 1 \equiv X_1 + X_2 + \dots + X_n \equiv $
	\item[] $\sum\limits_{X_k \equiv 1( mod 2 )} X_k$ + $\sum\limits_{X_j \equiv 0( mod 2 )} X_j$
	\item[] Desaparece la sumatoria de los pares por la congruencia.
	\item[] $\equiv \sum\limits_{X_k \equiv 1( mod 2 )} 1$
	\item[] $\equiv t$ $ \equiv $ {Cantidad de pilas impares} 
\end{itemize}

Por lo tanto, por transitividad de la congruencia tenemos que {Cantidad de pilas impares} $\equiv 1$
\end{proof}

Por lo enunciado anteriormente, la suma nim de la posicion nim inicial, sera \textbf{distinta de cero}, que es lo que queriamos ver.