\subsection{Estrategias ganadoras del jugador II}
Para los incisos a) y b) de este ejercicio, definiremos ciertas cosas a continuacion.

\textbf{Nota}: $\bigoplus$ es la suma nim.\\

\begin{theorem}[Teorema de Bouton]
\label{bouton}
	Una posicion $(X_1, \dots, X_n)$ en Nim, es \texttt{P-Position} $\leftrightarrow$ $X_1 \bigoplus ... \bigoplus X_n = 0$ 
\end{theorem}

\begin{theorem}
\label{particion-movidas}
	Las posiciones de un juego se particionan en \texttt{N-Position} y \texttt{P-Position}.
\end{theorem}

\begin{theorem}[Propiedad caracteristica de las posiciones]
	\label{position-characterization}
	Las posiciones de tipo \texttt{N-Position} y \texttt{P-Position} se definen recursivamente como:
	\begin{enumerate}
		\item Todas las posiciones terminales son \texttt{P-Position}.
		\item Desde toda \texttt{N-Position} \textbf{existe al menos una} jugada a una \texttt{P-Position}
		\item Desde toda \texttt{P-Position}, \textbf{todas} las jugadas son hacia \texttt{N-Position}.
	\end{enumerate}
\end{theorem}

\textbf{Idea:} Comenzando en una \texttt{N-Position}, en cada turno del jugador II, este siempre debe moverse a una posicion en la cual el jugador I vuelva a mover a una \texttt{N-Position}.\\

\begin{theorem}[Estrategia ganadora de II]
	\label{estrategia-ganadora}
	Si el jugador II comienza en una \texttt{N-Position}, tiene una estrategia ganadora.	
\end{theorem}
\begin{proof}
	Si el jugador II comienza en una \texttt{N-Position}. Por definicion(\ref{position-characterization}), existe una jugada que me lleva a una \texttt{P-Position}.
	Por definicion de \texttt{P-Position}, el jugador I va a tener que perder el juego porque dicha posicion es terminal o moverse a una \texttt{N-Position}, y nuevamente, el jugador II puede moverse a alguna \texttt{P-Position}. Este proceso se repite y es la estrategia ganadora para el jugador II.\\	
\end{proof}

\begin{corollary}
	\label{estrategia-comienza-I}
	Por definicion de \texttt{P-Position}(\ref{position-characterization}), todas las movidas posibles llevan a una \texttt{N-Position}. Si el jugador I comienza en una \texttt{P-Position}, el jugador II tambien tiene estrategia ganadora desde el momento que I le cede la \texttt{N-Position} al mover o pierde porque es una posicion terminal.
\end{corollary}

\begin{corollary}
	\label{sumanim-posiciones}
	Usando los teoremas \ref{bouton} y \ref{particion-movidas} tenemos que si $(X_1, \dots, X_k)$ es el nim inicial, entonces vale:
	\begin{itemize}
		\item La posicion inicial es una \texttt{P-Position} si $(X_1 \bigoplus \dots \bigoplus X_k) = 0$ 
		\item La posicion inicial es una \texttt{N-Position} si $(X_1 \bigoplus \dots \bigoplus X_k) \neq 0$ 
	\end{itemize}
\end{corollary}

\begin{theorem}[Equivalencia de suma nim a un bitwise exclusive or]
\label{equiv-nim-xor}
	La suma nim es equivalente a aplicar un \textbf{bitwise XOR} a los numeros a y b en base 2. Es decir, podemos definir la suma nim como $a \bigoplus b$ $ = $ $( (a)_2$ $\hat{}$ $(b)_2 )_{10}$.
\end{theorem}
\begin{proof}
	 Se desprende de considerar las tablas de definicion de la suma modulo 2 y la tabla de verdad del xor tomando 0 como falso y 1 como verdadero.
\end{proof}

\subsubsection{Inciso a}
Sabemos que:
\begin{itemize}
	\item El numero n, de fichas en la bolsa, es par
	\item El numero k, de pilas, es par 
\end{itemize}

Dado que n es par, luego de la etapa de colocacion de fichas, comienza jugando el nim el jugador I.\\
Como establecimos en \ref{estrategia-comienza-I} y \ref{sumanim-posiciones}. Si el nim inicial $(X_1, \dots, X_k)$ es tal que $(X_1 \bigoplus \dots \bigoplus X_k) = 0$, el jugador I comenzara en una \texttt{P-Position} y el jugador II tendra la estrategia ganadora mencionada en \ref{estrategia-ganadora}.\\

Solo resta encontrar una manera para que el nim inicial sume cero.\\

\textbf{Idea:} Le balanceo las pilas nim al jugador I de a pares. Luego, en el nim inicial $(X_1, \dots, X_k)$, tengo una cantidad par k de torres tales que:\\
\begin{enumerate}
	\item \textbf{Para cada pila, existe otra con la misma cantidad de elementos:} $\forall $ $ X_j $ $ \in $ $(X_1, \dots, X_k)$ $\exists$ $ X_i $ tal que $ (i \neq j) \land (X_i = X_j) $  
	\item \textbf{Ley de cancelacion de suma nim:} $ (X_i \bigoplus X_j) = 0$  
\end{enumerate}

Por lo tanto, el nim inicial que comienza jugando I es una \texttt{P-Position}. Como queriamos.

\begin{theorem}[Se puede armar un nim inicial que suma cero]
	Bajo estas hipotesis, siempre se puede llegar a un nim balanceado de a pares.
\end{theorem}

\begin{proof}
Por Induccion en cantidad de fichas:\\

Sean:
\begin{itemize}
	\item El numero n, de fichas en la bolsa, natural par no nulo.
	\item El numero k, de pilas, natural par no nulo.
	\item P(n): {Las pilas estan balanceadas(misma cantidad de elementos) de a pares}.
\end{itemize}

Dado que la cantidad de pilas es par, podemos establecer una relacion de apareo entre pares de pilas. Diremos que 2 pilas estan apareadas cuando tienen la misma cantidad de elementos. Concretamente, al final de cada par de turnos tendremos una cantidad par de pares de pilas apareadas.

\begin{itemize}
	\item \textbf{Caso base, n = 2}\\
	Como k es par(y no nulo), entonces $k \geq 2 $. El jugador I coloca una ficha en alguna de las pilas. Luego, existe alguna pila vacia. El jugador II pone una ficha en la pila vacia. Finalmente para toda pila en el nim, o bien esta vacia, o bien es alguna de las 2 pilas con 1 elemento.

	\item \textbf{Caso inductivo, n $\rightarrow$ n+1}
	Vale la hipotesis inductiva. El jugador I coloca una ficha en alguna pila $X_j$. Como las pilas estan balanceadas, existe una pila $X_i$, \texttt{apareada} a $X_j$ tal que si el jugador II coloca una ficha en ella quedan $X_j$ y $X_i$ con la misma cantidad de fichas, balanceadas nuevamente. Como no tocamos ninguna otra pila. Las pilas vuelven a quedar balanceadas.

\end{itemize}
\end{proof}

\subsubsection{Inciso b}
Sabemos que:
\begin{itemize}
	\item El numero n, de fichas en la bolsa, es impar
\end{itemize}

Dado que n es impar, juega primero el jugador II en el nim, queremos que II comience en una \textbf{N-Position}.
Equivalentemente por \ref{sumanim-posiciones}, queremos que la posicion inicial $(X_1, \dots, X_k)$ sume nim \textbf{distinto de cero}, $(X_1 \bigoplus \dots \bigoplus X_k) \neq 0$.

\begin{theorem}
Sea $(X_1, \dots, X_k)$ una posicion en el nim, esta posicion es una \texttt{P-Position} $\leftrightarrow$ $(X_1 \bigoplus \dots \bigoplus X_k) = 0$, usando \ref{equiv-nim-xor} tenemos que:\\

\vspace{0.3cm}

\begin{tabular}{|c|c c|}
			\hline
			& $X_1 	=$ & $	(X_{1,1}, \dots, X_{1,t})_2$ \\\hline
			& $X_2 	=$ & $	(X_{2,1}, \dots, X_{2,t})_2$ \\\hline
$\bigoplus$  & & $ \dots$\\\hline
			& $X_k 	=$ & $	(X_{k,1}, \dots, X_{k,t})_2$ \\\hline 
			& $0$	 & $	(0,   ..., 0)_2$ \\\hline
\end{tabular}
\vspace{0.3cm}

\textbf{Nota:} $t \in \mathbb{N}$ es la cantidad de bits requerida para representar el maximo numero del conjunto ${ X_1, \dots, X_k}$.\\

Esto ocurre unicamente cuando para todo $j$ entre $1$ y $t$ vale que $X_{1,j} + X_{2,j} + \dots + X_{k,j} = 0 $ $ (mod$ $2)$. Es decir, cuando la cantidad de unos en cada una de las columnas es par.\\
\end{theorem}

\begin{proof}
	Es facil ver esto, considerando la asociatividad del XOR en la suma por columna mencionada anteriormente, aplicando la operacion xor de a 2 elementos por columna hasta que finalmente da 0 cuando la cantidad de unos es par, 1 caso contrario.
\end{proof}

\begin{corollary}
	Como nosotros queremos comenzar a jugar el nim en una \textbf{N-Position}, debe existir \textbf{ALGUNA} columna tal que la operacion exclusive or en dicha columna de distinto de cero, es decir, que dicha columna tenga una cantidad \textbf{IMPAR} de unos.
\end{corollary}

\begin{lemma}
	El bit mas a la derecha en la suma nim especificada como operacion exclusive or, indica el bit de paridad del numero $X_j$, es decir. Si dicho bit es 0, el numero $X_j$ es par, caso contrario, el numero $X_j$ es impar.
\end{lemma}

\textbf{Idea:} Conseguir una cantidad impar de torres con cantidad impar de fichas en el armado de las torres.\\

\begin{theorem}[]
	Al ser $n$ impar, el numero de columnas $X_j$ en el nim inicial $(X_1, \dots, X_k)$ tales que $X_j$ es impar,tambien lo sera.
\end{theorem}

\begin{proof}

Sabemos que $n = X_1 + X_2 + \dots + X_n $ y n es impar.

\begin{itemize}
	\item[] Tomando congruencia modulo 2.
	\item[] $n \equiv 1 \equiv X_1 + X_2 + \dots + X_n \equiv $
	\item[] $\sum\limits_{X_k \equiv 1( mod 2 )} X_k$ + $\sum\limits_{X_j \equiv 0( mod 2 )} X_j$
	\item[] Desaparece la sumatoria de los pares por la congruencia.
	\item[] $\equiv \sum\limits_{X_k \equiv 1( mod 2 )} 1$
	\item[] $\equiv t$ $ \equiv $ {Cantidad de pilas impares} 
\end{itemize}

Por lo tanto, por transitividad de la congruencia tenemos que {Cantidad de pilas impares} $\equiv 1$
\end{proof}

Por lo enunciado anteriormente, la suma nim de la posicion nim inicial, sera \textbf{distinta de cero}, que es lo que queriamos ver.

\subsubsection{Inciso c}
Sabemos que:
\begin{itemize}
	\item k, el numero de pilas del nim, es 3. 
\end{itemize}

Trabajaremos con una tupla $(x, y, z)$. Pero cualquier permutacion es analoga ya que no nos importa el orden de las pilas.\\

\begin{proof}
Probaremos esto usando induccion , en la cantidad de turnos, bajo el siguiente predicado:\\
P(n): {I puso n fichas y el nim satisface que (x, y, z) tal que $x = y$, $z \geq y$}\\

\begin{itemize}
	\item Caso base: 
		P(1): I coloco una ficha en el nim vacio. Y este quedo de la forma (0, 0, 1). Lo cual cumple con la restriccion pedida.
	\item Caso inductivo: 
		$P(n) \rightarrow P(n+1):$
		Podemos suponer por la hipotesis inductiva que vale lo siguiente:
		\begin{itemize}
			\item El nim tiene la forma (x, y, z) tal que x=y, $z \geq y$.
			\item El jugador I coloco n fichas y el proximo turno es de II.
		\end{itemize}

		Tenemos 3 casos:
		\begin{enumerate}
			\item Si II coloca (x, y, z+1), en el proximo turno n+1-esimo, I contesta con (x, y, z+2) lo cual satisface lo que queremos.
			\item Si II coloca (x+1, y, z), en el proximo turno, (n+1) I puede contestar (x+1, y+1, z) si $z>y$ o (x+1, y, z+1) si z=y.
			\item Si II coloca(x, y+1, z) es analogo al caso anterior(b), pues x=y por HI.
		\end{enumerate}

		Luego, $P(n) \rightarrow P(n+1)$ pues:
		\begin{itemize}
			\item El nim tiene la forma (x, y, z) pedida.
			\item El jugador I coloco n+1 fichas y el proximo turno es de II.
		\end{itemize}
		
\end{itemize}
\end{proof}

\subsubsection{Inciso d}

\begin{lemma}[Caracterizacion de potencias de 2 en binario]
	Todo numero de la forma $n = 2^k$, $ k \in \mathbb{N}$ Se escribe en base 2 como $(1, \dots, 0)_2$ donde, de derecha a izquierda, hay k ceros en su tira de bits y luego un uno.
\end{lemma}

\begin{lemma}[Caracterizacion de potencias de 2 menos uno en binario]
	\label{bin-pot2menos1}
	Todo numero de la forma $n = 2^k - 1$, $ k \in \mathbb{N}$ Se escribe en base 2 como $(0, 1, \dots, 1)_2$ donde, de derecha a izquierda, hay k unos en su tira de bits.
\end{lemma}

\begin{theorem}
Si $X = (2^h - 1)$, $h \in \mathbb{N}$, entonces $X \bigoplus (X+1) = (2^{h+1} - 1)$.\\
\end{theorem}
\begin{proof}
	Utilizando la equivalencia de la suma nim, con la operacion de bits xor y las caracterizaciones de los lemas anteriores. Es inmediato ver que el resultado de la operacion $X \bigoplus (X+1)$ = $(1, \dots, 1)_2$, donde de derecha a izquierda hay k+1 unos. Usando la caracterizacion del lema \ref{bin-pot2menos1}, vemos que este numero es $2^{h+1}-1$, como queriamos ver.
\end{proof}

\begin{theorem}
Si $X \neq (2^h - 1)$, $h \in \mathbb{N}$, entonces $X \bigoplus (X+1) < X $.\\
\end{theorem}
\begin{proof}
Vemos que como $X \neq (2^h - 1)$ entonces existe un primer cero de derecha a izquierda tal que $X = (1, \dots, 0, 1, \dots)_2 $ donde los primeros puntos suspensivos indican la posible presencia de bits genericos y los segundos puntos suspensivos indican la posible presencia de unos.\\
Luego, tenemos que $X + 1 = (1, \dots, 1, 0, \dots)_2 $ donde los primeros puntos suspensivos indican los mismos bits que en X y los segundos puntos suspensivos indican la posible presencia de ceros.\\
Luego tenemos que $X \bigoplus (X+1) = (0, \dots, 0, \dots)_2$ Las posiciones en el desarrollo binario donde X y X+1 eran prefijos, quedan en 0 por la operacion de XOR. Luego, es claro ver que $X \bigoplus (X+1) < X$.
Para ver esto, consideremos que $(0_{j}, \dots, 0, \dots)_2$ es un numero claramente menor que $2^{j}$.
Pero en el desarrollo en base 2 de X, aparece dicho bit en 1. Con lo cual $ X \geq 2^{j} $. Luego por transitividad, $X \bigoplus (X+1) < X$.
\end{proof}

\subsubsection{Inciso e}
Sabemos que:
\begin{itemize}
	\item n es par.
	\item $n \neq 2^j - 2$.
	\item $k = 3$ 
\end{itemize}

Queremos que I tenga estrategia ganadora. Esto es equivalente a decir que queremos que I comience en una \textbf{N-Position} o que II comience en una \textbf{P-Position}. Usando los teoremas definidos anteriormente, esto es equivalene a que sea $(X_1, \dots, X_n)$ el nim inicial, su suma nim de cero o distinto de cero, segun querramos.

Usando lo demostrado en el \textbf{inciso c} sabemos que I tiene forma de dejar el nim inicial de la forma $(x, x, y)$ con $y \geq x$.\\

\textbf{n par:} II pone la ultima ficha y el primer turno del nim es de I\\
Tomando lo visto en la demo del inciso c, II puede dejar el nim inicial, de las siguientes maneras:
\begin{enumerate}
	\item Si II coloca (x, x, y+1) con $x \leq y$, luego el nim inicial suma distinto de cero pues $x \bigoplus x \bigoplus (y+1) = 0 \bigoplus (y+1) = y+1 \neq 0$. I Comienza en una \textbf{N-Position}, por lo tanto tiene estrategia ganadora.

	\item Si II coloca (x, x+1, y), con $x \leq y$.
		\begin{enumerate}
			\item Si $x = y$, entonces $ x \bigoplus y \bigoplus (x+1)= 0 + (x+1) = (x+1) \neq 0$. y es identico al caso anterior.
			\item Si $ y = x+1$, entonces tenemos que $ x \bigoplus (x+1) \bigoplus (x+1) = x$. Si $x \neq 0$ comenzaremos en una \textbf{N-Position} que es lo que queremos. Veamos que x no puede ser cero, por el absurdo: Supongamos que x=0. Entonces $n = 0 + 1 + 1 = 2 = 2^2 - 2$, lo cual es absurdo por hipotesis.  
			\item Si $ y > x+1$, entonces queremos ver como queda $x \bigoplus (x+1) \bigoplus y$.
			 Como vimos en el inciso d, tenemos dos casos nuevamente:
			 \begin{enumerate}
				 \item $x \bigoplus (x+1) < x$ si $x \neq 2^k -1$:\\
				 Como $x < y$, entonces $((x+1) \bigoplus x )< x < y$, mas particularmente nos sirve el hecho de que $((x+1) \bigoplus x ) \neq y$, porque ahora sabemos que $((x+1) \bigoplus x )\bigoplus y\neq 0$ y I tiene estrategia ganadora.
				 \item $x \bigoplus (x+1) = 2^{k+1} -1$ si $x = 2^k -1$:\\
				 I tiene una estrategia ganadora $\leftrightarrow$ $x \bigoplus (x+1) \bigoplus y \neq 0$ $\leftrightarrow$ $2^{k+1}-1 \bigoplus y \neq 0 $ $\leftrightarrow$ $2^{k+1}-1 \neq y $. Demo: Supongamos que $y = 2^{k+1}-1$, luego $n = x + (x+1) + y$ $\leftrightarrow$ $n = (2^k -1) + (2^k -1 +1) + (2^{k+1}-1)$ $\leftrightarrow$ $n = (2^{k+1} -1) + (2^{k+1}-1)$ $\leftrightarrow$ $n = (2^{k+2}-2)$, lo cual es absurdo por la hipotesis de que $n \neq 2^j -2$. 

			 \end{enumerate}
		\end{enumerate}
	\item el caso donde II coloca (x+1, x, y) es analogo por conmutatividad.
\end{enumerate}